\section{In Scope / Out of Scope}
\label{sec:In Scope / Out of Scope}

A 'scope' is an abstract term for setting the area or subject relevant for the discussion.

\begin{quote}
For example, two men meet for a coffee. For sure, they will start to discuss business issues, and only business issues will be in scope of their conversation. Any other stuff (cars, pets, women) will be 'out of scope'.

But when this two will have to exterminate the second bottle of \index{Whiskey}'Johnnie Walker', business issues will become 'out of scope', and chat will be oriented to cars, women, pets and other important for business issues.                                                                                                                                                                                                                     \end{quote} 

Same thing relate to testing: testers have to establish from the start which functionality will be tested during next testing session; they should list all functions 'in scope' and all the functions, which will not be tested during this session, and agree with Development and Client about why this will be and those will not be tested.

\begin{quote}
By the way, such documents are called Testing Scope and they are an indispensable part of any \index{Test Plan}Test Plan.\end{quote} 

Why testers must define the Testing Scope? This should be done especially to cover testers heads in case of strange questions appears, like '\textit{But why this and those things wasn't tested?}'. 

Sometimes testers can save their lives with this 'in scope' and 'out of scope' lists.

Sometimes not.