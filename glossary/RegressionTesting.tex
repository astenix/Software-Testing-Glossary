\section{Regression Testing}
\label{sec:Regression Testing}

This process is very annoying and costs are high, but we can drive it in a smart way, trough \rindex{\textbf{I}!Impact Analysis}Impact Analysis ([p.\pageref{sec:Impact Analysis}]).

\rindex{\textbf{R}!Regression Testing}So, every system (social, political, military or software system) should continuously expand his functional possibilities to survive. This is a \textbf{Progress}.

But with more opportunities, the more relationships appears, and the more chances to find a defect. Or not to find it. This is a \textbf{Regress}.

\textbf{Regression testing} is performed for getting know, if regression appears.

\begin{quote}
 Testing brings information, remember this simple idea?
 \end{quote} 

This activity only looks like we should test some modification in software to ensure that defects have not been introduced or uncovered in unchanged areas of the software, as a result of the changes made. In fact, 'Regression testing' can be done periodically, without any changes in software. 

\begin{quote}
Now deal with it.\end{quote} 

One of our interest is to discover negative behavior, that's why we are doomed to perform a 'Regression testing' each time when new functionality is added to a software.

And 'Regression testing' it is sometimes performed when the environment is changed.

\begin{quote}
 Environment, not software. You can call this 'Integration testing', but what kind of Integration it is, if the software still the same?
\end{quote} 

You will be a great donkey, if you will advance in 'Regression testing' having only old \rindex{\textbf{T}!Test Case}test cases, '\textit{Cause this is just a re-test, right, ma?}'. You will need new test cases because of ANY significant change in software structure.
