\section{Acceptance Testing}
\label{sec:Acceptance Testing}

A very ambiguous term. Depending of context:

\textbf{1}

A development project phase.

At the end of any development project phase a customer representatives can started a (sometimes~\textemdash~\emph{very}) rigorous checking of every requirements, expectations and business capabilities implemented during the development process. Sometimes it is the unique solution to prevent a possible brilliant technical solution from failure.

\index{Customer}Customer did not test, customer \emph{use} the product. You \emph{test} the product. You cannot test the product as a Customer does it.

\begin{quote}
You can test till the death on the \textbf{Development} and \textbf{Testing} environments, but only the \russianq{\textbf{UAT}} (\index{UAT, User Acceptance Testing}User Acceptance Testing) environment can be considered as the closest to production environment. So, some serious bugs can be founded \textbf{only} on UAT environment.                                                                                                                                                                                                                                                                                                                                             \end{quote} 

That's why \russianq{Acceptance Testing} can be done only from the customer point of view~\textemdash~by the customer himself. We can only help him, by sharing our test-docs/ideas.

\begin{quote}
And only by the Customer itself. 

Or only by his brave testers and accountants.
\end{quote}

Only on UAT you can use for checkout not only 'Visa for testing' cards, but real payment issues.

Only on UAT you can use real payment filters, or a real database with customer's stuff, or a real connection with Warehouse.

And this is the reason to not invite development team to test on UAT and Production servers.

\textbf{2}

An approach to software development.

This approach is coming from TDD ('Testing Driven Development') family.

Implies that every requirement statement is published and handled by the customer as automated Test case (using with special software like \index{FitNesse}'FitNesse' wiki-system - see demo at \url{https://youtu.be/wzmVJ3HYftA}.
