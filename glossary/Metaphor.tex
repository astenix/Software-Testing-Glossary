\section{Metaphor}
\label{sec:Metaphor}

A \index{Metaphor}\emph{Metaphor} is a figure of speech that refers, for rhetorical effect, the attributes of one thing by mentioning another thing.

It may provide clarity or identify hidden similarities between two ideas. A \textit{Metaphor} directly equates this two items, and does not use "like" or "as" as does a \textit{Simile}. 

Example\footnote{~Yeah, the word "like" was used, but anyway}: 

\begin{verse}
The moon has looming over the horizon like a big orange. 
\end{verse} 

From the other side, nobody will understand the same metaphor, used vice versa:

\begin{verse}
An orange has lying on a plate, like a moon over the horizon. 
\end{verse} 

Sound stupid, isn't it?

Some methapors works perfect in both sides, some not.
