\section{Root Cause Analysis}
\label{sec:Root Cause Analysis}

\rindex{\textbf{R}!Root Cause Analysis}A root cause is an initiating cause of a causal chain which leads to an outcome or effect of interest. In short: \textit{find why someone was killed, and you can find the killer}.

Root cause analysis is a method of problem solving that tries to identify the root causes of faults or problems. A lot of developers can expect such analyse from testers, but this is insane. Root cause analysis practice solve problems by attempting to identify and correct the root causes of events, as opposed to simply addressing their symptoms. Focusing correction on root causes has the goal of \emph{preventing problem recurrence}, and here testers fails.

There are many different tools, processes, and philosophies for performing Root Cause Analysis. However, several very-broadly defined approaches or "schools" can be identified by their basic approach or field of origin: safety-based, production-based, process-based, failure-based, and systems-based.
