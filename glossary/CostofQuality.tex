\section{Cost of Quality}
\label{sec:Cost of Quality}

\rindex{\textbf{C}!Cost of Quality}This is a calculation of the total cost of quality-related efforts and deficiencies. The results of such calculations sometimes lead to sever managerial decisions at the whole Project scale.

It was first described by \rindex{\textbf{A}!Armand Feigenbaum}Armand Feigenbaum in a 1956 Harvard Business Review article.

Cost of Quality is the total costs incurred on quality activities and issues and often split into prevention costs, appraisal costs, internal failure costs and external failure costs.

If you don't know how much \$\$ costs the last founded Defect, then you cannot know the Cost of Quality on your Project.

Advanced in the Project, this amount grows, because with the time each Project engage more and more software, time and materials. Sometimes managers try to reduce the cost of quality without reduce the quality. Well, sometimes this help a lot. Sometimes this is not the smartest decision. It depends of...

Check the 'Cost of poor quality' ideas \url{https://en.wikipedia.org/wiki/Cost_of_poor_quality}.
