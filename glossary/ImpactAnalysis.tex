\section{Impact Analysis}
\label{sec:Impact Analysis}

\index{Impact Analysis}In fact, this should be named \textbf{Change impact analysis}, but anyway~\textemdash~this is a method to identify the potential consequences of a change in a complex system.

Consider \index{Notepad}'Notepad' as a simple software under test. We can add new functions (or modify existed). We would like to set assure, that this change will not alter existed functions. How?

We can proceed to regression testing (consider the level of boring testers and costs). Or we can setup a table, where we will trace all Notepad functionality and their junction points.

Now, we can see, which functionality will be affected by changes, and can retest only areas, where changes will have the impact.

Problem with this impact tables is the same as for any other project documentation~\textemdash~they should be updated on a regular basis.

And yes, if the project is short or 'small'~\textemdash~we can omit such documents.
