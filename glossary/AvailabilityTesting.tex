\section{Availability Testing}
\label{sec:Availability Testing}

\index{Availability Testing}\emph{Availability} is the probability that a system will work \textbf{as} required \textbf{when} required during the period of a mission. 

You won't have to check it often in functional testing. Let's define it just for the sake of understanding.

\textbf{Main idea}: to reduce and eliminate the expected downtime, you should have a clear knowledge of how much time the app (or the machine) will be operational till the death, if nobody intervenes.

This is very important to know when we talk about web-servers (or aircrafts, or hard drives and so on).

Testing for availability means 
\begin{itemize}
\item 
running an application for a planned period of time, 
\item 
collecting 
	\begin{itemize}
	\item 
	failure events 
	\item
	and repair times, 
	\end{itemize}
\item 
and comparing the availability percentage to the original service level agreement.                                                                                                                                                                                                                  \end{itemize}

Availability testing is primarily concerned with measuring and minimizing the actual repair time. This is a simple mathematics. 

But before of all you should know several things:

I. $Mean~Time~To~Failure$ (MTTF) 

\begin{quote}
Is the time, on average, that you would expect a clock-work to fail when it has been running. It is a simple indicator of  clock-work \index{Reliability}reliability.

Example: a clock-work runs for 500 hours. It breaks down 5 times during that period.

$Mean~Time~To~Failure$ (MTTF) $=$ 500 $/$ 5 $=$ \textbf{100 hours}
\end{quote} 

II. $Mean~Time~Between~Failure$ (MTBF) 

\begin{quote}
Is the time, on average, that you would expect a clock-work to fail including time lost for repairs are undertaken. It is an indicator of the combined reliability and \index{Maintenance efficiency}maintenance effectiveness/efficiency.

Example: a clock-work runs for 500 hours, and it has 200 hours downtime due to 5 failures.

$Mean~Time~Between~Failure$ (MTBF) $=$ (500 $+$ 200) $/$ 5 $=$ \textbf{140 hours}                               \end{quote} 

III. $Mean~Time~To~Repair$ (MTTR) 

\begin{quote}
Is the time, on average, that you would expect a stoppage to last including time spent waiting for maintenance engineer, diagnosis, waiting for parts, actual repair and testing. 

It is an indicator of maintenance effectiveness/efficiency. 

Note that there is another indicator that can be used here, $Mean~Corrective~Repair~Time$ (MCRT). 
\begin{quote}
$Mean~Corrective~Repair~Time$ is only interested in the actual repair time assuming all tools, spares and required manpower are available (efficiency).

$Mean~Time~To~Repair$ (MTTR) $-$  $Mean~Corrective~Repair~Time$ (MCRT)  $=$  Waste and therefore gives you an indication of how ineffective your stores and maintenance resource strategy is.                                                                                                                                                                                                                                                                                                                                                                                                                                                                                                                                                                                                        \end{quote} 

Notice that as MTTR trends towards zero, the percentage availability trends towards 100\%.

\end{quote}


Still here? Here is an example.

As above, clock-work runs for 500 hours, has 200 hours downtime due to 5 failures.

$Mean~Time~To~Repair$ (MTTR)  $=$  200  $/$  5  $=$  \textbf{40 hours}

Therefore: 

$Mean~Time~To~Failure$ (MTTF)  $+$  $Mean~Time~To~Repair$ (MTTR) $=$  $Mean~Time~Between~Failure$ (MTBF)

Now, how to calculate the \emph{Availability} — just use data from following variables:

\begin{enumerate}

 \item $Mean~Time~To~Failure$ (MTTF)
 \item $Mean~Time~Between~Failure$ (MTBF)
 \item and $Mean~Time~To~Repair$ (MTTR)
\end{enumerate}

Or more importantly, the unavailability of the clock-work due to maintenance causing failures.

Example, huh?

As above, clock-work runs 500 hours breaks down 5 times and 200 minutes are spent waiting repair $/$ spares $/$ repairing.

Availability:
\begin{quote}

 $=$  ((500 $+$ 200) $-$ 200) $/$ (500 $+$ 200)

 $=$  (700 $-$ 200) $/$ 700

 $=$  71\%
\end{quote}

Or

\begin{equation}\label{eq:Availability calculation}
\frac{Mean~Time~To~Failure~(MTTF)}
{Mean~Time~Between~Failure~(MTBF)} =  Availability
\end{equation}

or 


\begin{equation}\label{eq:Same Availability calculation}
\frac{Mean~Time~To~Failure~(MTTF)}
{Mean~Time~To~Failure~(MTTF)  +  Mean~Time~To~Repair~(MTTR)}  =  Availability
\end{equation}

\begin{quote}
100 $/$ 140  $=$  71\% 
\end{quote}

Or

\begin{quote}
100 $/$ (100 $+$ 40)  $=$  71\%
\end{quote}

Ok. \textbf{What about \emph{Unavailability}?}

\begin{quote}
1  $-$  $Availability$  $=$  29\%
\end{quote}

Or
\begin{quote}
200  $/$  (500  $+$  200)  $=$  29\%
\end{quote}

Or

\begin{quote}
MTTR $/$ MTBF or MTTR $/$ (MTTF $+$ MTTR)

40 $/$ 140 or 40 $/$  (100  $+$ 40)  $=$  29\%
\end{quote}


Still here?

Here is an another example of the formula for calculating percentage availability: 
\begin{equation}\label{eq:Percentage Availability}
\frac{Mean~Time~Between~Failures}{Mean~Time~Between~Failures  +  Mean~Time~To~Repair} \times 100
\end{equation}

"\textit{Mean Time}" means, statistically, the average time.

"\textit{Mean Time Between Failures}" is literally the average time elapsed from one failure to the next.  Usually people think of it as the average time that something works until it fails and needs to be repaired (again).

"\textit{Mean Time To Repair}" is the average time that it takes to repair something after a failure.