\section{Entry and Exit Criteria for Testing}
\label{sec:Entry and Exit Criteria for Testing}

\begin{quote}
Understand what is a \index{Criteria}Criteria first [p.\pageref{sec:Criteria}].
\end{quote} 

Sometimes is very hard to agree about when we can start or stop testing.

For example, developers already had developed something, and \index{Customer}Client call us to start testing. In his opinion, this is enough for starting testing~\textemdash~here are some functionality, just use your imagination and common sense and go on. But we still have no requirements, and we cannot be sure, that we are on the same page with developers about what and how should work. Can we just shut up and let's the mortal testing begin, or we will hang on our hands and refuse to start, because we had no time for preparation, no test cases, no requirements\ldots?

Yes, if we have common sense and we can discuss with our customer any issue, we can start without any requirements and fears.

But if the Customer is \index{Mafia}Al Capone itself, and in case of something will fail on our website you will wake up in the morning being silently buried in a black box under the heavy ground\ldots We need to agree something before the process will start.

Entry/exit Criteria for Testing are a set of generic and specific Conditions discussed between the client and executor. This criteria doesn't exist as a law, and always can be/should be revisited.

\subsection{Entry criteria for testing}
\label{sec:Entry criteria for testing}

\index{Entry criteria}The purpose of Entry criteria is to prevent a task from starting which would entail more (wasted) effort compared to the effort needed to remove the failed entry criteria.

Entry criteria for testing can be following:

    \begin{itemize}
\item 
for each functionality we have a requirement available,
\item 
    we have enough human resources for the testing process,
\item 
    all third-party units are available for being used with our system under test,
\item 
    build is frozen and no new functionality will be added or changed...                                                                        \end{itemize}


\subsection{Exit criteria for testing}
\label{sec:Exit criteria for testing}

\index{Exit criteria} The purpose of Exit criteria is to prevent a task from being considered completed when there are still outstanding parts of the task which have not been finished. Exit criteria are used to report against and to plan when to stop testing. 

Exit criteria for testing can be following:

    \begin{itemize}
\item     all critical functionality was tested and here is the complete report,
\item     all discovered critical bugs was fixed and retested,
\item     the time for testing is out...\end{itemize}

By the way, sometimes \textbf{Exit Criteria} for Testing is called as \textbf{Completion Criteria}.
