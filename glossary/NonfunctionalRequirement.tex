\section{Non-functional Requirement}
\label{sec:Non-functional Requirement}

A \index{Requirements}requirement that does not relate to functionality, but to attributes such as reliability, efficiency, usability, maintainability and portability.

Typically non-functional requirements fall into areas such as:
\begin{itemize}
\item 
Accessibility
\item     Capacity, current and forecast
   \item  Compliance
    \item Documentation
  \item   Disaster recovery
    \item Efficiency
    \item Effectiveness
    \item Extensibility
   \item  Fault tolerance
    \item Interoperability
   \item  Maintainability
   \item  Privacy
   \item  Portability
   \item  Quality
  \item   Reliability
   \item  Resilience
   \item  Response time
    \item Robustness
   \item  Scalability
  \item   Security
   \item  Stability
  \item   Supportability
   \item  Testability               \end{itemize}

\begin{quote}
How you will test the Effectiveness of \index{MS Word}MS Word processor? :)
\end{quote} 

Non-functional requirements are sometimes defined in terms of metrics (something that can be measured about the system) to make them more tangible. For example, the Effectiveness of MS Word processor can be described binary~\textemdash~it is effective or not. Or it can be described, for example, from 1 to 10 points.

Suppose the level of user excitement when he drive an Cadillac Eldorado 1959~\textemdash~this is a non-functional requirement.

Suppose the level of user annoying when he use an app, which crashes each time when wi-fi connection is down for a second~\textemdash~this is a non-functional requirement.

\subsection{Non-functional Testing}
\label{sec:Non-functional Testing}

Testing the attributes of a component or system that do not relate to functionality, e.g. reliability, efficiency, usability, maintainability and portability.
