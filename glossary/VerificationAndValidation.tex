\section{Verification}
\label{sec:Verification}

This is the basic level of any testing, it comes from the verb \texttt{to verify}.

\rindex{\textbf{V}!Verification}Verification is the simplest confirmation of any \emph{expectation} (or assumption) by examination and through provision of objective evidence that specified requirements have been fulfilled.

\textbf{Example}: have a pie. 

\begin{quote}
Or a \emph{borscht}. 

Or a car.
\end{quote}

Verification is the basic checking that this pie can be eaten, and if so, then the quality of the pie is ok.

Sometimes it is more than enough just to be sure, that software does what was expected.

But sometime not.

I am sure, that you will expect, that the pie should be tasty, right?!

\subsection{and Validation}
\label{sec:and Validation}

\rindex{\textbf{V}!Validation}Confirmation by examination and through provision of objective evidence that the informal requirements for a specific intended use or application have been fulfilled. 

Any pie can be cooked as expected by recipe, and this can be tested (verified). But even if recipe is only one, the taste of pies will vary from one cook to other. And because it is very hard to set the requirements of 'how to do a tasty pie', this start to lead to informal requirements. And such reqs can be tested only at the \emph{Validation} level.

This two terms are always presented in testing, but they are not declared. 

Any \rindex{\textbf{T}!Test Case}test case can be declared as '\textit{This is a validation test}' or '\textit{This test is about verification only}'. But obviously, nobody will ask '\textit{Let's write some validational test cases, please}'.
