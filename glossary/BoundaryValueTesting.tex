\section{Boundary Value Testing}
\label{sec:Boundary Value Testing}

A test-design technique for explain the source where tester can search for ideas to create test cases.

Represent an input (or output) value which is on the edge of an equivalence partition or at the smallest incremental distance on either side of an edge (boundary).

\begin{description}
\item[Example:] 
       You can test a car at his minimum and maximum speed values (10 and 100 miles per hour). 
       
       If the test will be ok, you can suppose, that in the interior of this range the car will work as expected. 
       
       So, instead of doing 90 test cases (10 m/h, 11 m/h, 12 m/h....) you can check only two test cases and suppose (just suppose) that car will work as expected.
\end{description}

Sometimes this approach can help to save time and materials.

Sometimes this approach totally sucks, if tester make a logical mistake or a false admit, that in the interior of the range everything will work as expected.