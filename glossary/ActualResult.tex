\section{Actual Result}
\label{sec:Actual Result}

You always have some \textbf{Expectations} when you make an action for testing. Otherwise the whole testing has no sense.

\begin{quote}
Testing is always to compare the \rindex{\textbf{E}!Expected result}\emph{Expected} and the \rindex{\textbf{A}!Actual result}\emph{Actual} results.

When you have no expectations, then you can make an investigation, a research~\textemdash~anything but not testing.
\end{quote}

Suppose, that I want to know if I can send an email notification from a Product page.

I have an expectation like \ordinaryq{\textit{I will call \ordinaryq{Send email} function, I will provide appropriate data in input fields, and I will receive a notification}}. This is the \textbf{Expected result} of a test case.

Suppose, that I am really open a Product Detail Page and I am looking for the \ordinaryq{Send email} function. It is available? If yes, than I really can send an email with it? 

This will be the \textbf{Actual result}.

You can have more than one actual results.

A tester should always suppose the \textbf{Expected result} (a mandatory part of any Test Case [p.\pageref{sec:Test Case}]), but he will never be sure about the Actual result.

If any difference between Expected and Actual Result~\textemdash~this can be a \hyperref[sec:Bug]{Bug} [p.\pageref{sec:Bug}].

\begin{quote}
Or the Expected result is wrong, and Test Case should be updated.

Or something was unexpectedly changed during development.

Or a third-party application has down. 
\end{quote} 
