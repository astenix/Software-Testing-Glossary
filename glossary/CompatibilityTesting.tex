\section{Compatibility Testing}
\label{sec:Compatibility Testing}

\index{Compatibility Testing}This is a part of non-functional testing. During this type of testing we try to understand, if the tested application is compatible with the computing environment or other specified application.

For example, an app designed to retrieve data from Internet should be compatible with all type of networks. Or an app, designed to work with Excel 97 format (.xls) should work with \index{MS Excel}Excel 2010 format (.xlsx)

Often you will hear about \index{Firefox}Browser compatibility testing, which can be more appropriately referred to as user experience testing, but anyway. This requires that the web applications are tested on different web browsers, to ensure the following:

\begin{itemize}
\item 
    Users have the same visual experience irrespective of the browsers through which they view the web application.
\item 
    In terms of functionality, the application must behave and respond the same way across different browsers and sometimes — operating systems.
\item 
    Hardware (different phones and applications).                                                 \end{itemize}
