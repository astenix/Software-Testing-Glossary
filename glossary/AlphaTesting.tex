\section{Alpha Testing}
\label{sec:Alpha Testing}

\index{Alpha Testing}Alpha Testing is the name of an approach to testing, where the Development team ships his work to \emph{internal} testing team only.

Get it? An \emph{approach}, not a phase in testing process. But anything can be presented like a phase, so\ldots

Alpha Testing approach is specific for development Internet Shops, games and almost any kind of software, where human beings opinion matters.

Surely, technically we can always publish an software with some bugs, and '\textit{…if users will tell us about it, we will fix them quickly. It is just a question of \index{Cost of Failure}'cost of failure}'. 

Well, today the cost of failure looks very low, because 'You can easily fix the software online'.

\begin{quote}
\index{Skype}Skype, \index{Firefox}Firefox, \index{Twitter}Twitter, \index{Facebook}Facebook~\textemdash~they always ship first, then fix.

Why don't you do the same?
\end{quote}

But 'Shops' always means 'Money', and money requires confidence. We cannot ship an Internet Shop with any bugs in functionality, because it can scarry future customers and this can burn out our asses. Or because they can cheat, and again, this is all about money.

And not every bug can be fixed quickly.

So we will frenzy test our software BEFORE it will be offered to our customers and will call this \emph{Alpha Testing}.

The next big step will/can be \index{Beta Testing}\textbf{Beta Testing} [p.\pageref{sec:Beta Testing}]: this means that the Development team will select a small bunch of potential users of their product (usually outside from the development company), and will ship to them the Product 'as is', just for testing purposes.

\begin{quote}
And, because you are asking, we are aware that there is no \emph{Gamma} or \emph{Delta} testing. 

But they was! An IBM PM Martin Belsky\footnote{~\href{http://bit.ly/2kf9S9H}{bit.ly/2kf9S9H}} has invented them just for name some logical steps.

\begin{description}
\item[The $A$-test] was a feasibility and manufacturability evaluation done before any commitment to design and development.

\item[The $B$-test] was a demonstration that the engineering model functioned as specified. 

\item[The $C$-test] (corresponding to today's beta) was the $B$-test performed on early samples of the production design.

\item[The $D$-test] was the $C$-test, repeated after the model had been in production a while, to verify final safety.                                                                                                               \end{description}

We don't care about this today, so\ldots
\end{quote}

There is a \textbf{Main problem} with Alpha Testing~\textemdash~internal testing cannot reveal EACH issue, that can happens in production. It can assure only that all functionality works as expected in expected scenarios and conditions.

\begin{quote}
Are you really sure, that all scenarios and conditions in your software was foreseen?

Really?
\end{quote} 

Often this limitation is very clear to development, but not for the \index{Customer}Customer.
