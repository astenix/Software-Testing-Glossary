\section{Automated Testing}
\label{sec:Automated Testing}

An approach to testing, where some functional testing in application under development can be done by special Robots instead of Human beings.

\begin{quote}
Sometime it helps. Sometimes not.

It can be fun. It can be stupid.\end{quote} 

Usually such testing activity can only demonstrate that your application works as expected~\textemdash~nothing was broken or damaged during development.

Automated test cases are very 'simple and stupid'.

\begin{quote}
Suppose, that you have a link as an image on the page, and the image was lost for some reason.

Automated tests will be 'blind and deaf' if no images will be available on page for a hyperlink~\textemdash~test script will click on link and the testing will go on without any error warnings.\end{quote} 

Automated test cases are very unstable, they will fail immediately, if the locators of elements on the page will have ANY (even one symbol) changes. For this reason, each fail should be investigated apart, because nobody knows, if the fail happens because of a bug, or because the test script itself is obsolete or incomplete.

Automated test cases run very fast, they can interact with tested software faster than any human can see.

\begin{quote}
For the first time this looks fun, but later you realize, that you cannot understand what happens on the page during testing. You lose the control.
\end{quote} 

Automated test cases can use an unlimited test data variety.

This approach involve a lot of special programming work before the testing begin and a lot of programming after.

This job can be done by developers or by skilled testers, but anyway~\textemdash~for being done, it requires the availability of well done developed and documented \index{Test Case}Test Cases.

The candidates for the test automation may be as follows by priorities:

 \begin{enumerate}
\item 
   test cases that are included in smoke testing;
\item 
test cases that are executed more frequently during regression testing;
\item 
test cases that are executed less often, such as low priority test cases that are not included either in smoke tests or in regression tests;
\item 
test cases that are difficult to execute manually because of large data sets, or because it takes too much time to setup and run).                                                                                                                                      \end{enumerate}
