\section{Beta Testing}
\label{sec:Beta Testing}

\rindex{\textbf{B}!Beta Testing}Beta Testing is a testing approach, specific only for mass-market products (\rindex{\textbf{W}!MS Word}Word, \rindex{\textbf{S}!Skype}Skype, \rindex{\textbf{F}!Firefox}Firefox so on).

Means that the Development team select a big bunch of already existed users of their product (usually this happens outside of the development company), and ship to them the Product in Testing 'as is'.

\textbf{Main idea}~\textemdash~developments will receive a lot of '\textit{Something is going wrong when I tried to do following...}' messages. This can help in testing the Product on different testing configuration, without spending time trying to create it in our testing laboratory. And User Experience can reveal something really unexpected.

\textbf{Main problem}: not every message from beta testing users is a \rindex{\textbf{B}!Bug Report}\emph{Bug Report}. Will be a lot of garbage or already reported issues. Will be generated a lot of issues, which cannot be reproduced for some reasons (usually, because of a lack of information and because bug reporters didn't know that there are some common rules of writing bug reports).

And even if all messages will be Bug Reports, do you know how to deal with a thousand of reports about the same \rindex{\textbf{B}!Bug}Bug? 

For internet shops Beta testing is not applicable at all (users can be scared of bugs at Checkout, so only Alpha Testing [p.\pageref{sec:Alpha Testing}] is appropriate), and is a part of standard testing process.
