\section{Version}
\label{sec:Version}

\index{Version}The versioning is the process of assigning either unique version names or unique version numbers to unique states of computer software. 

Usually the version of a program is presented like this: 

\begin{quote}
ver. \texttt{3.12.242}\end{quote} 

Let's read it:

\begin{description}
 \item[Major release] — \#3. 
 
 The software was developed to the third release.
 
 \item[Minor release] — \#12.
 
 In the third major release was added some functionality. Maybe was added 12 new functions in one day. 
 
 Maybe not, maybe to 11 already existed functions today was the 12. 
 
 Maybe in major release \#3 was added only one new function, with 11 important updates.

 \item[Build] — \#242 \index{Build}
 
 I can suppose, that someone had a bad day and made 242 bugfixes. Each bugfix increase build numbering with +1.
\end{description}

At a fine-grained level, revision control is often used for keeping track of incrementally different versions of electronic information, whether or not this information is computer software.

We can change numbers at any time. Now we have ver. \texttt{3.12.242}. Add one new feature, and this can be ver. \texttt{3.13.0} Apply some small modifications, and we have ver. \texttt{3.13.1} Add one new feature, and here is the ver. \texttt{3.14.0} 