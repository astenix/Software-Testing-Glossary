\section{Agile Testing}
\label{sec:Agile Testing}

Even if it should be based on \index{Agile Software Development}Agile Software Development principles\footnote{\url{http://agilemanifesto.org/}} [p.\pageref{sec:Agile Software Development}], the Testing activities still remain the same.

There are no such thing like \index{Agile Testing}\emph{Agile Testing}.

\begin{quote}
What is the difference between drinking water from a cup or from a bottle? Anyway, you just drink water.
\end{quote}

The same with Testing — you should have to do all old things like 

\begin{enumerate}
 \item 
 gather requirements,
 \item 
 understand expextations,
  \item 
 understand and preview situations for being tested,
  \item 
 the expected results
  \item 
  …
 \end{enumerate}

The most common problem started immediately from '\emph{There are no \index{Requirements}Requirements}' issue.

Everything else in Agile Testing derive from this goddamned '\emph{there are no Requirements}' mantra. You can have \index{User Story}User stories \& \index{Use Case}Use Cases, but for writing brilliant \index{Test Case}Test Cases you would scream for good brand old \index{Requirements}Requirements.

The Testing process in agile development requires a totally new logical approach, and you may not fit with it at all.

For example, for a tester in Agile is very important to be a master in \index{Exploratory Testing}Exploratory Testing or at least \index{Ad hoc testing}Ad Hoc testing, not a master in writing and executing \index{Test Case}Test Cases; and is important to handle very well the \index{Automated Testing}Automated Testing (from unit to complex functional testing), for increase speed of testing.

Some person can excellent fit, others will certainly fail — this is a matter of psychology.

And you can fail because you act with Agile like with just another \emph{development process}, not like with an approach of do the development \emph{closest to client needs}, and this is the second most common problem with Agile testing. 