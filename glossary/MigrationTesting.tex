\section{Migration Testing}
\label{sec:Migration Testing}

\rindex{\textbf{M}!Migration Testing}Consider that a company switch from one DBMS to another (from MySQL to Oracle). Or the architecture of DB requires a massive innovation. Then will be required a database Migration testing.

Database migration may be done manually, but it is more common to use an automated ETL (Extract-Transform-Load) process to move the data.

Database migration testing may encounter problems when:
\begin{enumerate}
 \item The data in the source database(s) changes during the test;
 \item Some of the source data is corrupt;
 \item The mappings between the tables/ fields of the source databases(s) and target database are changed by the database development/ migration team;
 \item A part of the data from the source database is rejected by the target database;
 \item Data migration takes too long because the database migration process is too slow or the source data file is too large.
 \end{enumerate}
 
 The test approach for database migration testing consists of the following activities:
 
\begin{enumerate}
 \item  
 Design the validation tests. In order to test database migration, SQL queries are created either by hand or using a tool, such as a query creator. The test queries should contain logging statements for the purpose of effective analysis and bug reporting after the tests are complete.
 \item Set up the test environment. The test environment should contain a copy of the source database, the ETL tool (if applicable) and a clean copy of the target database. The test environment should be isolated so that it is not changed externally during the tests.
 \item Run your validation tests Depending on the test design, the database migration process does not need to completely finish before starting tests.
 \item Report the bugs.
 \end{enumerate}
