\section{Ad hoc testing}
\label{sec:Ad hoc testing}

The term \rindex{\textbf{A}!Ad hoc testing}\ordinaryq{\emph{ad hoc}} has a Latin language origin, and can be translated like 

	\begin{quote}
	\ordinaryq{\textit{Emergency, jump on everybody, there is no time to think, we have no plan, just let's kick out some barbarians, just go-go-go!}}.
	\end{quote} 

For us this is a name of a testing approach, usually explained as testing, but without test cases or requirements\footnote{~See figure \ordinaryq{Simple explanation of Classic \& Ad Hoc \& Exploratory Testing} at [p.\pageref{fig:ClassicAdHocExploratoryTesting}].}:

\begin{itemize}
\item without formal test preparation;
\item with no recognized test design technique is used;
\item without expectations for results and arbitrariness guides the test execution activity. 
\end{itemize}

But this is very, very stupid!

In fact, nobody said \russianq{DO NOT USE REQUIREMENTS in Ad hoc testing!}
\begin{quote}
Use them!
\end{quote} 

Nobody said \russianq{DO NOT USE TEST CASES in Ad Hoc testing!}
\begin{quote}
Use them!
\end{quote} 

The Ad Hoc is the second part of Classic Testing. Like 
\begin{itemize}
\item 
White \& Black, 
\item 
Guitar \& Strings, 
\item 
Bonnie \& Clyde.
\end{itemize}

The Ad hoc testing approach aims to \emph{support} the formal testing approach (you call it Classic testing), where all testing activities are based on formal and logical execution.

The weakest side of formal testing is the lack of some scenarios, that can be omitted by development team, but they can be discovered by end-users. The Ad Hoc testing can help to identify possible bugs, that cannot be discovered using formal scenarios.

Some monkeys think that the Ad Hoc testing is the same as \rindex{\textbf{M}!Monkey Testing}'Monkey Testing' [p.\pageref{sec:Monkey Testing}]. They're wrong.

\begin{itemize}
\item {In monkey testing tester even doesn't understand, what and for what he has doing with the application.}

\item {In Ad Hoc testing tester understand his goals and application capabilities, he implies logic in his actions, but he can easily switch between his goals and scenarios, without any reason for doing it.}
\end{itemize}
