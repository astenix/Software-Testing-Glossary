\section{Functional Requirement}
\label{sec:Functional Requirement}

\index{Functional Requirement}In programming a Function is a named section of a program that performs a specific task.

In the same way that a File is a named section of some data on your hard-disk drive.

\begin{quote}
The term function is also used synonymously with operation and command. For example, you execute the 'delete' function to erase a word.                                                                                                                                       \end{quote} 

A \textbf{Function}, in its most general use, is what a given entity does in being what it is ([p.\pageref{sec:Function}]).

The \textbf{Functionality} is an ability of software to perform a task.

A \textbf{Functional Requirement} is an abstract, which describes the functionality task that a software system should \textbf{do} for the user needs.

\begin{quote}
And a Non-functional requirements usually place constraints on \textbf{how} the system will do so. See [p.\pageref{sec:Non-functional Requirement}] for details.                                                                                        \end{quote} 

An \textbf{example} of a functional requirement would be that a system must send a an email whenever a certain condition is met (e.g. an order is placed, a customer signs up, etc).

\begin{quote}
A related non-functional requirement for the system can be that emails should be sent with a latency of no greater than 12 hours from such an activity.
\end{quote} 

Then \textbf{Functional Requirements} are done, it's time to write \textbf{Functional Specifications}.

You understand the difference between Requirements and Specifications?