\section{White Box / Black Box Testing}
\label{sec:White Box Black Box Testing}

It is completely not about colors (Black/White), but nobody cares (and clearly will not).

Please, do not share the stupid idea that \ordinaryq{\textit{White Box can be done when you look into the source code, and the Black Box can be done when you don't have an acces to source code}}. There are thousand of situations, when you'll look directly into the source code, and you will perform classic Black Box testing.

\begin{quote}
 Later you will realize, that developers understand this \ordinaryq{boxes} better than testers\ldots
\end{quote} 

\rindex{\textbf{W}!White Box} \rindex{\textbf{B}!Black Box}\ordinaryq{White~\&~Black} boxes are just a very convenient \emph{metaphors} (see p.\pageref{sec:Metaphor}) in testing terminology. They perfectly explain the source where tester has searched for ideas to create \rindex{\textbf{T}!Test Case}test cases\footnote{~Situations to be tested}.

Keep it simple.  As a tester you have to 
\begin{itemize}
\item 
realize what kind of situation may happen when user will start to interact with the application (and which situations can happen, but should not), 
\item 
create these situations one by one and see what's happen.                                                                                                                                                                                                              \end{itemize}

This is the whole \emph{Testing}.

You can have all those situations listed in \rindex{\textbf{R}!Requirements}Requirements (p.\pageref{sec:Requirement}). If it so, then you are a lucky bastard (or you are in army). But for a mass-market application (any modern website) you will be out of this luxury. So, you have to invent or discover them by yourself. How to do it?

Sometimes you can know exactly how your app should working. You will imagine situations (so \rindex{\textbf{T}!Test Case}test cases will appear), and you will swtich to the illuminated zone~\textemdash~the \ordinaryq{White box} metaphor.

\begin{quote}
Doesn't matter, where you will bring this information. You can read requirements? Now you know. You can read source code? Now you know. You can ask someone who knows about it? Now you know too. 
\end{quote} 

Sometimes you know nothing, and you slide to a darker zone (the 'Black box' metaphor), where you have to often suppose what should happen than knowing exactly. You should explore, like \ordinaryq{Let's provide to this input field several characters and see how the software will react to it\ldots}.

As you can see, the \emph{researching} (or, sometimes, exploration) is not a testing. They are always close, but they are not the same.

\begin{quote}
In testing you ALWAYS know the \rindex{\textbf{E}!Expected result}expected result. And you can compare it with \rindex{\textbf{A}!Actual result}actual result.

In researching you may be blind about the expected result, but when you will know~\textemdash~you can test. That's why texting and researching are always close.
\end{quote} 

\subsection{Why \ordinaryq{boxes}}

The software, in the eyes of a tester, is like a box with some magic inside.

You don't know exactly how and why the coffe-machine works, but you can interact with it. The \ordinaryq{box} is \ordinaryq{black}.

You may not understand how and why an engine works (or how it looks like), but you can drive a car. The \ordinaryq{box} is \ordinaryq{black}.

You don't know what to do with your miserable life, but you still live it. The \ordinaryq{box} is \ordinaryq{black}.

And if you know How and Why it works~\textemdash~the box became \ordinaryq{transparent} for you.

\begin{quote}
That's why someone can say \ordinaryq{A glass box}, or even \ordinaryq{The transparent box}, opposite to \ordinaryq{The black box}.
\end{quote} 

\subsection{What about strategies}

\rindex{\textbf{G}!Glenford J. Myers}Bearded old school testers (covered by bearded old school developers) claims that the \ordinaryq{White/Black Box Testing} is a \emph{strategy}\footnote{~\russianq{Art of Software Testing} by Glenford J. Myers, 1979, John Wiley \& Sons, Inc.} (see  p.\pageref{sec:Strategy}). And if this is a strategy, then we can combine the \ordinaryq{black} and \ordinaryq{white} to the \ordinaryq{gray box}.

Well, it really looks alike. But no.

And there are no \ordinaryq{gray} boxes in testing.
