\section{Smoke Testing}
\label{sec:Smoke Testing}

\index{Smoke Testing}This is just a convenient metaphor, defining a process of superficial testing, required to being done before the real, deeper advanced testing started.

The idea consist in following: before starting to explore new build, set assure that all at least main functionality (differ from project to project) is on:

    \begin{itemize}
\item 
product catalog is available,
\item     you can add a product to cart,
\item     you can run trough checkout process,
\item     user profile are available,
\item     and so on.              \end{itemize}

If smoke testing will fail, then the whole testing process will be worthless.

If smoke testing is ok, then we can advance in testing, as expected.

Usually developers call this checking \index{Sanity Testing}'sanity testing', others (factory engineers) \index{Integrity Testing}'integrity testing'. As manual functional testers, we called it 'Smoke testing'.

\begin{quote}
A lot of years ago the testing of electronic devices started with real 'smoke testing'~\textemdash~plug the device to an outlet and set power 'on'.

If you see light and smoke~\textemdash~forget about testing, the device is broken, you will burn.                                                                        \end{quote} 

Smoke testing is always based on a subset of defined/planned \index{Test Case}test cases, that cover the main functionality of a component or system, to ascertaining that the most crucial functions of a program work, but not bothering with finer details.

A daily build and smoke test is among industry best practices.
